\documentclass{beamer}

% Theme
\usetheme{Madrid}
\usecolortheme{default}

% Packages
\usepackage{graphicx}
\usepackage{longtable}
% Title Page
\title{Develop a 2D Occupancy Grid Map of a Room using Overhead Cameras}
\author{Alphonsa Abraham,Navneeth Krishnan,Sidharth V Menon,Vyshnavi Dipu}
\institute{Saintgits Group of Institutions}
\date{\today}

\begin{document}

\begin{frame}
  \titlepage
\end{frame}

\begin{frame}
  \frametitle{Introduction} 
  \begin{itemize}
    \item The use of overhead cameras for indoor mapping and monitoring has become important due to their ability to provide comprehensive, real-time views of a room to detect and track object to identify the occupied and unoccupied spaces within the room.
    \item This project focuses on developing a 2D occupancy grid map of a room using overhead cameras.This approach has wide-ranging applications, in robotics for navigation, in smart building management for optimizing the space usage, and in security systems for monitoring unauthorized access.
  \end{itemize}
\end{frame}

\begin{frame}
  \frametitle{Unique Idea Brief}
   \begin{enumerate}
   \item Phase 1 : Static Environment Mapping
   \begin{itemize}
    \item Arrange four overhead RGB cameras in a 2x2 pattern with overlapping fields of view.
    \item Capture and stitch images from the cameras and merge the images into  a single  panoramic view.
    \item Convert the stitched images into a 2D occupancy grid map, marking occupied and free spaces.
    \end{itemize}
    \item Phase 2 : Dynamic Environment Mapping
    \begin{itemize}
    \item Continuously capture and process images to dynamically update the occupancy grid map as objects move.
    \item Capturing images at 1s intervals and updating the positions using cv2's capture method works without the need for a model to be trained and semantic labels. It automatically adjusts the image that is sent to the Occupancy Grid Map generator to update the Grid Map
     \end{itemize}
    \end{enumerate}
\end{frame}

\begin{frame}
  \frametitle{Unique Idea Brief}
  \begin{enumerate}
        \item Simulation and Comparison in Gazebo
    \begin{itemize}
    \item Add an AMR equipped with an onboard camera or LiDAR in the Gazebo environment and Implement SLAM using the ROS2 navigation stack to generate an occupancy grid map from the AMR's perspective.
    \item Compare the map generated by the overhead cameras with the map created by the AMR using SLAM.
  \end{itemize}
  \end{enumerate}
\end{frame}

\begin{frame}
  \frametitle{Process Flow}
    \begin{itemize}
      \item Setup Cameras: Install and calibrate four overhead RGB cameras in a 2x2 grid.
      \item Capture and Stitch Images: Simultaneously capture images, detect features, and stitch them into a panoramic view.
    \item Generate Occupancy Grid: Convert to grayscale, apply thresholding, and create a 2D occupancy grid map.
      \item Real-time Updates: Continuously capture images, detect changes, and update the grid map in real-time.
    \end{itemize}
\end{frame}

\begin{frame}
  \frametitle{Process Flow}
    \begin{itemize}
 \item Object Detection: Implement object detection and add semantic labels
    \item Gazebo Setup: Model the room, objects, and cameras in Gazebo.
      \item AMR SLAM Mapping: Add AMR with onboard camera use ROS2 for SLAM
      \item Compare Maps: Compare overhead camera map with AMR’s SLAM map for accuracy and effectiveness.
       \end{itemize}
\end{frame}
\begin{frame}

  \frametitle{Architectural Diagram }
  \begin{figure}
      \centering
      \includegraphics[width=1\linewidth]{images/Untitled Diagram.png}
      \caption{Caption}
      \label{fig:enter-label}
  \end{figure}
\end{frame}

\begin{frame}
  \frametitle{Technologies used}
  \begin{itemize}
   \item Ubuntu 20.04 Focal Fossa
   \item ROS2 Foxy
   \item Turtlebot3
   \item Rviz
   \item OpenCV(cv2)
   \item Gazebo
  \end{itemize}
\end{frame}

\begin{frame}{Evaluation}
\begin{itemize}
    \item The main component used in evaluation was Rviz and we used it to compare distances between key points on the map of the house provided.
    The measurements came out to be as below:
\end{itemize} 

\begin{longtable}{|l|l|}
\hline
Connection & Grid map Length \\ \hline
\endfirsthead
%
\multicolumn{2}{c}%
{{\bfseries Table \thetable\ continued from previous page}} \\
\hline
Connection & Grid map Length \\ \hline
\endhead
%
1          & 4.84m           \\ \hline
2          & 5.63m           \\ \hline
3          & 6.84m           \\ \hline
4          & 7.24m           \\ \hline
5          & 9.28m           \\ \hline
6          & 6.86m           \\ \hline
7          & 5.76m           \\ \hline
8          & 3.73m           \\ \hline
\caption{}
\label{tab:my-table}\\
\end{longtable}

\end{frame}


\begin{frame}
  \frametitle{Team members amd contributions}
  \begin{itemize}
   \item Navneeth Krishnan and Vyshnavi Dipu : Gazebo simulation environment, Initial image acquisition and stitching simulation,Integration with ROS2 Navigation stack
   \item Alphonsa Abraham and Sidharth V Menon : Real-time Image Processing Simulation,Object Detection and Semantic Labeling Simulation,Comparison and Analysis Simulation,Final Documentation and Report 
  \end{itemize}
\end{frame}

\begin{frame}
  \frametitle{Conclusion}
  \begin{itemize}
    \item This project successfully demonstrated the creation of a 2D occupancy grid map using simulated overhead cameras in a virtual environment. By integrating image stitching, real-time image processing, and object detection simulations, we achieved a dynamic map that reflects changes in the environment.
    \item Comparison with a simulated AMR's SLAM-generated map highlighted the map's accuracy and suitability for path-planning and navigation tasks. 
    \end{itemize}
\end{frame}
\end{document}
